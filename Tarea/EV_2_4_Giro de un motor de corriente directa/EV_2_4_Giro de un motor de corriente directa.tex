\documentclass[11pt,a4paper]{article}
\usepackage{graphicx}
\usepackage{amsmath}
\usepackage{amssymb}
\usepackage{mathrsfs}
\usepackage{cancel}

\begin{document}
\begin{center}
\textbf{REPORTE DE TAREA}\\
GIRO DE UN MOTOR DE CORRIENTE DIRECTA
\end{center}

\begin{center}
Maria de Lourdes Gomez Islas\\
15-OCT-2019\\
Universidad Politecnica de La Zona Metropolitana de Guadalajara
\end{center}

\part{Introduccion}
Los motores electricos de corriente continua son el tema de base que se amplia en el siguiente trabajo, definiendose en el mismo los temas de mas relevancia para el caso de los motores electricos de corriente continua, como lo son: su definicion, los tipos que existen, su utilidad, distintas partes que los componen, clasificacion por excitacion, la velocidad, la caja de bornes y otros mas.\\
Esta maquina de corriente continua es una de las mas versatiles en la industria. Su facil control de posición, par y velocidad la han convertido en una de las mejores opciones en aplicaciones de control y automatización de procesos. Pero con la llegada de la electronica su uso ha disminuido en gran medida, pues los motores de corriente alterna, del tipo asincrono, pueden ser controlados de igual forma a precios mas accesibles para el consumidor medio de la industria. A pesar de esto los motores de corriente continua se siguen utilizando en muchas aplicaciones de potencia (\emph{trenes y tranvias}) o de precision (\textbf{maquinas, micro motores, etc.})

\section{Motor de corriente directa}
Un motor electrico de Corriente Continua es esencialmente una maquina que convierte energia electrica en movimiento o trabajo mecanico, a traves de medios electromagneticos.

\part{Utilización de los motores de corriente directa}
Se utilizan en casos en los que es importante el poder regular continuamente la velocidad del motor, ademas, se utilizan en aquellos casos en los que es imprescindible utilizar corriente directa, como es el caso de motores accionados por pilas o baterías. Este tipo de motores debe de tener en el rotor y el estator el mismo numero de polos y el mismo numero de carbones.\\
LOS MOTORES DE CORRIENTE DIRECTA PUEDEN SER DE TRES TIPOS:\begin{itemize}
\item SERIE
\item COMPOUND
\item PARALELO
\end{itemize}

\section{MOTOR SERIE:}
Es un tipo de motor electrico de corriente continua en el cual el devanado de campo (campo magnetico principal) se conecta en serie con la armadura. Este devanado esta hecho con un alambre grueso porque tendra que soportar la corriente total de la armadura.

Debido a esto se produce un flujo magnético proporcional a la corriente de armadura (carga del motor). Cuando el motor tiene mucha carga, el campo de serie produce un campo magnético mucho mayor, lo cual permite un esfuerzo de torsión mucho mayor.\\
Sin embargo, la velocidad de giro varía dependiendo del tipo de carga que se tenga (sin carga o con carga completa). Estos motores desarrollan un par de arranque muy elevado y pueden acelerar cargas pesadas rápidamente.

\section{MOTOR PARALELO:}
es un motor de corriente continua cuyo bobinado inductor principal esta conectado en derivacion con el circuito formado por los bobinados inducidos e inductor auxiliar.\\
Al igual que en las dinamos shunt, las bobinas principales están constituidas por muchas espiras y con hilo de poca sección, por lo que la resistencia del bobinado inductor principal es muy grande.

\section{MOTOR COMPOUND:}
es un motor de corriente continua cuya excitacion es originada por dos bobinados inductores independientes; uno dispuesto en serie con el bobinado inducido y otro conectado en derivacion con el circuito formado por los bobinados inducido, inductor serie e inductor auxiliar.\\
Los motores compuestos tienen un campo serie sobre el tope del bobinado del campo shunt. Este campo serie, el cual consiste de pocas vueltas de un alambre grueso, es conectado en serie con la armadura y lleva la corriente de armadura.


\end{document}