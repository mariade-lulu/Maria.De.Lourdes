\documentclass[11pt,a4paper]{article}
\usepackage{graphicx}
\usepackage{amsmath}
\usepackage{amssymb}
\usepackage{mathrsfs}
\usepackage{cancel}

\begin{document}
\begin{center}
\textbf{REPORTE PRACTICA}\\
.\\
PUENTE H\\
\end{center}

\begin{figure}[h]
\centering
\includegraphics[width=12.5cm]{upzmg.png} 
\end{figure}

\begin{center}
Maria de Lourdes Gomez Islas\\
.\\
10-OCT-2019\\
Universidad Politecnica de La Zona Metropolitana de Guadalajara
\end{center}

\newpage 

\part{INTRODUCCION}
Un Puente en H es un circuito electronico que generalmente se usa para permitir a un motor electrico DC girar en ambos sentidos, avance y retroceso. Son ampliamente usados en robotica y como convertidores de potencia. Los puentes H están disponibles como circuitos integrados, pero tambien pueden construirse a partir de componentes discretos.\\
El termino "\textbf{puente H}" proviene de la tipica representacion grafica del circuito. Un puente H se construye con 4 interruptores (mecanicos o mediante transistores).

\section{CIRCUITO}

\begin{figure}[h]
\centering
\includegraphics[width=9cm]{HTML/PUENTEH.png} 
\caption{Puente H}
\end{figure}

\subsection{PARTE PRINCIPAL}
De acuerdo con el circuito anterior "\textbf{Optoacopladores y relevadores}" haremos funcionar un puente H conectandole los 5V de salida de ese circuito al del circuito siguiente de la figura 1.

\newpage 

\subsection{FUNCIONAMIENTO}
Pusimos nombre a cada interruptor o transistor para hacer mas facil la practica con una tabla de verdad:\\

\begin{figure}[h]
\centering
\includegraphics[width=10cm]{HTML/wwwwwww.png} 
\caption{PTabla de verdad}
\end{figure}

\section{CONCLUSION}
Como hemos dicho el puente H se usa para invertir el giro de un motor, pero tambien puede usarse para frenarlo (de manera brusca), al hacer un corto entre las bornas del motor, o incluso puede usarse para permitir que el motor frene bajo su propia inercia, cuando desconectamos el motor de la fuente que lo alimenta.

\end{document}