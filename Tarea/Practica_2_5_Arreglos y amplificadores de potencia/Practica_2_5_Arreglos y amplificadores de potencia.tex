\documentclass[11pt,a4paper]{article}
\usepackage{graphicx}
\usepackage{amsmath}
\usepackage{amssymb}
\usepackage{mathrsfs}
\usepackage{cancel}

\begin{document}
\begin{center}
\textbf{REPORTE PRACTICA}\\
.\\
ARREGLOS DE AMPLIFICADORES DE POTENCIA
\end{center}

\begin{figure}[h]
\centering
\includegraphics[width=13cm]{upzmg.png} 
\end{figure}

\begin{center}
Maria de Lourdes Gomez Islas\\
.\\
17-OCT-2019\\
Universidad Politecnica de La Zona Metropolitana de Guadalajara
\end{center}


\newpage

\part{INTRODUCCION}
Una de las funcionalidades mas importantes de un transistor es la de amplificar senales. tambien podemos hacer cambios de bases electromecanicas , con estos cambios es posible hacer muchos tipos de senales como tambien negarlas\\
Los reguladores de potencia mas sencillos son lineales. Existen dos tipos de circuitos integrados (CI) aptos para esta función: los amplificadores lineales y los reguladores de tension lineales.\\
Los transistores bipolares de potencia se pueden emplear tanto en aplicaciones lineales como en conmutacion, aunque son más lentos y sensibles al fenomeno de la segunda ruptura, el cual es el resultado de una distribución no uniforme de la corriente en la union base-colector (polarizada inversamente durante conduccion) del transistor de salida, provocando un aumento de la temperatura en aquella zona que puede destruir el dispositivo; y que es distinto de la ruptura primaria por avalancha.

\section{Circuitos}
En orcad hicimos 4 circuitos donde pudimos observar la ganancia de cada amplificador, el \emph{SUMADOR}, \emph{RESTADOR}, \emph{NO INVERSOR} e \emph{INVERSOR}.
Donde para sacar la ganancia del amplificador \textbf{inversor} era:\\
 $$ \frac{RF}{R1} $$\\
Y la formula para saber la ganancia de un amplificador \textbf{no inversor}:\\
$$ \frac{RF}{R1} + 1 $$\\

\newpage 
\section{Simulacion en ORCAD}
\begin{figure}[h]
\centering
\includegraphics[width=9cm]{HTML/SUMADOR.png} 
\caption{SUMADOR}
\end{figure}

\begin{figure}[h]
\centering
\includegraphics[width=9cm]{HTML/RESTADOR.png}  
\caption{RESTADOR}
\end{figure}

\newpage 

\section{Simulacion en PSpice}

\begin{figure}[h]
\centering
\includegraphics[width=9cm]{HTML/SUMADOR2.png} 
\caption{SUMADOR}
\end{figure}

\begin{figure}[h]
\centering
\includegraphics[width=9cm]{HTML/RESTADOR2.png} 
\caption{RESTADOR}
\end{figure}

\newpage 

\section{CIRCUITOS ADC}

\begin{figure}[h]
\centering
\includegraphics[width=7cm]{HTML/CIRCUITOADC.png} 
\caption{ADC}
\end{figure}

\begin{figure}[h]
\centering
\includegraphics[width=7cm]{HTML/ADCFOCOS.png} 
\caption{ADC-FOCOS}
\end{figure}

\newpage 

\section{Tablas de verdad}

\begin{figure}[h]
\centering
\includegraphics[width=4cm]{HTML/tabladeverdad1.png}   
\caption{Tabla de verdad SW}
\end{figure}

\begin{figure}[h]
\centering
\includegraphics[width=3.5cm]{HTML/tabladeverdadfocos.png} 
\caption{Tabla de verdad focos}
\end{figure}

\newpage 

\section{conclusion}
Para hacer las tablas de verdad era importante tomar como recomendacion el poner un voltaje de entrada segun la formula indicada en la practica, y variar el voltaje de salida para saber con cuanto coltaje se prenderia el foco y respecto al primer circuito era impotante tomar en cuenta el voltaje que daba en orden a la tabla de verdad, el cual nos dimos cuenta que el voltaje aumentaba.

\end{document}