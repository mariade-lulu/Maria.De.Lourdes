\documentclass[11pt,a4paper]{article}
\usepackage[latin1]{inputenc}
\usepackage[spanish]{babel}
\usepackage{amsmath}
\usepackage{amsfonts}
\usepackage{amssymb}
\usepackage{graphicx}
\usepackage{fourier}
\usepackage[left=2cm,right=2cm,top=2cm,bottom=2cm]{geometry}
\author{Maria de Lourdes Gomez}

\begin{document}
\begin{center}
\textbf{REPORTE DE TAREA}\\
.\\
CALCULAR LOS PARAMETROS DE CIRCUITOS DE ACTIVACION DE TRANSISTORES DE POTENCIA\\
.
\end{center}

\begin{center}
Maria de Lourdes Gomez Islas\\
.\\
29-OCT-2019\\
.\\
Universidad Politecnica de La Zona Metropolitana de Guadalajara
\end{center}

\begin{figure}[h]
\centering
\includegraphics[width=14cm]{upzmg.png} 
\end{figure}

\newpage 

\part{que es un transistor de potencia}

El funcionamiento y utilizacion de los transistores de potencia es identico al de los transistores normales, teniendo como caracteristicas especiales las altas tensiones e intensidades que tienen que soportar y, por tanto, las altas potencias a disipar.\\
Existen tres tipos de transistores de potencia:\\

\begin{itemize}
\item bipolar.
\item unipolar o FET (Transistor de Efecto de Campo).
\item IGBT.
\end{itemize}

\begin{figure}[h]
\centering
\includegraphics[width=9cm]{HTML/TAREAPOTENCIATRANSISTOR.png} 
\caption{Parametros}
\end{figure}

El IGBT ofrece a los usuarios las ventajas de entrada MOS, más la capacidad de carga en corriente de los transistores bipolares:\\
\begin{itemize}
\item Trabaja con tension. 
\item Tiempos de conmutacion bajos. 
\item Disipacion mucho mayor (como los bipolares).
\end{itemize}\\
Nos interesa que el transistor se parezca, lo mas posible, a un elemento ideal:\\
\begin{itemize}
\item Pequenas fugas. 
\item Alta potencia.
\item Bajos tiempos de respuesta (ton , toff), para conseguir una alta frecuencia de funcionamiento. 
\item Alta concentracion de intensidad por unidad de superficie del semiconductor. 
\item Que el efecto avalancha se produzca a un valor elevado ( VCE maxima elevada). 
\item Que no se produzcan puntos calientes (grandes di.dt ).
\end{itemize}\\
Una limitacion importante de todos los dispositivos de potencia y concretamente de los transistores bipolares, es que el paso de bloqueo a conduccion y viceversa no se hace instantaneamente, sino que siempre hay un retardo (ton , toff). Las causas fundamentales de estos retardos son las capacidades asociadas a las uniones colector - base y base - emisor y los tiempos de difusion y recombinacion de los portadores.\\

\end{document}