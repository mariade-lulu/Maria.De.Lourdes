\documentclass[11pt,a4paper]{article}
\usepackage[latin1]{inputenc}
\usepackage[spanish]{babel}
\usepackage{amsmath}
\usepackage{amsfonts}
\usepackage{amssymb}
\usepackage{graphicx}
\usepackage{fourier}
\usepackage[left=2cm,right=2cm,top=2cm,bottom=2cm]{geometry}
\author{Maria de Lourdes Gomez}

\begin{document}
\begin{center}
\textbf{REPORTE DE TAREA}\\
.\\
DISENO DE UNA MODULACION DE ANCHO DE PULSO (PWM) AMP.OP CON TRASISTORES\\
.
\end{center}

\begin{figure}[h]
\centering
\includegraphics[width=14cm]{upzmg.png} 
\end{figure}

\begin{center}
Maria de Lourdes Gomez Islas\\
.\\
22-OCT-2019\\
.\\
Universidad Politecnica de La Zona Metropolitana de Guadalajara
\end{center}

\newpage 

\part{Teorico}

Los circuitos de conversion DC/AC tienen amplia aplicacion en la industria.  Son  utilizados  en  varia-dores de velocidad, sistemas de ali-mentacion  ininterrumpida,  filtrosactivos,  etc.\\
Los  conversores  DC/AC  se  clasifican  como  inversores con fuente de voltaje (VSI) e inver-sores  con  fuente  de  corriente(CSI).\\
1 ,2 Los CSI se usan en siste-mas de alta potencia, los VSI se re-servan para aplicaciones en baja y mediana  potencia.  Dentro  de  estaclasificacion existen varias configu-raciones de conversores DC/AC quedependen de la aplicacion final y el nivel  de  voltaje  o  corriente  de  su salida.\\
En el caso de los drive para motores  de  baja  y  mediana  poten-cia, la topologia tipica es el medio puente inversor trifasico con fuentede voltaje , formado por seis  elementos  de  conmutacion Mosfets, Transistores Bipolares de Compuerta Aislada (IGBT), Tiris-tores  desactivados  por  Compuerta(GTO) o Tiristores Controlados porMOS (MCT).

\begin{figure}[h]
\centering
\includegraphics[width=16cm]{HTML/QWER.png}  
\caption{Medio puente inversor trifasico con circuito intermedio de DC.}
\end{figure}

\part{Tecnicas de modulacion escalares o PWM}

Se usa en inversores  DC/AC monofasicos y trifasicos.\\
Se basanen la comparacion de una senal dereferencia  a  modular  y  una  senalportadora  de  forma  triangular  odiente de sierra ; la com-paracion generará un tren de pulsos de ancho espesifico que se utilizanen la conmutacion del puente inver-sor.\\
La relacion entre la amplitud dela senal portadora y la senal de re-ferencia se llama \emph{índice de modu-lacion} y  se  representa  por  ,  donde  Ar  es  la  amplitud  de  la senal de referencia y Ac es la am-plitud de la senal portadora.\\
El indice  de  modulacion  permite  obtener tension variable a la salida delinversor.\\
La relacion entre la frecuencia dela senal portadora y la frecuencia de referencia  se  denomina  \emph{indice  defrecuencia}.

\end{document}