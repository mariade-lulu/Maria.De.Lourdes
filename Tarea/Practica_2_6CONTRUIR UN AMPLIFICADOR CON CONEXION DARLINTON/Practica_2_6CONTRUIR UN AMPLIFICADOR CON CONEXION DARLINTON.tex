\documentclass[11pt,a4paper]{article}
\usepackage{graphicx}
\usepackage{amsmath}
\usepackage{amssymb}
\usepackage{mathrsfs}
\usepackage{cancel}

\begin{document}
\begin{center}
\textbf{REPORTE PRACTICA}\\
.\\
CONTRUIR UN AMPLIFICADOR CON CONEXION DARLINTON
\end{center}

\begin{figure}[h]
\centering
\includegraphics[width=13cm]{../upzmg.png}  
\end{figure}

\begin{center}
Maria de Lourdes Gomez Islas\\
.\\
10-OCT-2019\\
Universidad Politecnica de La Zona Metropolitana de Guadalajara
\end{center}


\newpage 

\part{INTRODUCCION}

En electronica, el transistor Darlington es un dispositivo semiconductor que combina dos transistores bipolares en una configuracion tipo Darlington en un unico dispositivo (\textbf{a veces llamado par Darlington}). Esta conexion permite que la corriente amplificada por el primer transistor ingrese a la base del segundo transistor y sea nuevamente amplificada.

\section{Funcionamiento}
El transistor Darlington es un tipo especial de transistor que tiene una alta ganancia de corriente. Está compuesto internamente por dos transistores bipolares que se conectan es cascada.\\
El transistor Darlington y su estructura interna .El transistor 1 entrega la corriente que sale por su emisor a la base del transistor 2.

\subsection{Materiales}
\begin{itemize}
\item Cables 
\item Resitencias o Potenciometro 
\item FFoto resistencia
\item Circuito Practica 2
\item Fuente de alimentacion 
\end{itemize}


\section{Procedimiento}
procedimos a conectar nuestra fuente el positivo hacia el circuito y negativo de nuestra tierra. el triac quedara con direccion hacia la patilla de en medio de nuestro tiristor, quedando en paralelo con el capacitor. nuestra resistencia quedara en serie con nuestro diodo, asi es la manera en la que colocamos en nuestro circuito. el resultado obtenido es como se platico en los parrafos anteriores es que cuando la foto resistencia sea alumbrada su resistencia sube y cuando esta no sea alumbrada su resistencia sera nula.

\newpage 

\subsection{CONCLUSION}
Para el éxito del acoplo y buen calculo del circuito en mi caso en el calculo desde la carga y la resistencia en paralelo al diodo que es de 2.2K.\\
Todos los amplificadores tienen un limite de amplificación ya sea por su fabricación.

\end{document}